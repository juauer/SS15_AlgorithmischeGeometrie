% !TEX TS-program = pdflatex
% !TEX encoding = UTF-8 Unicode

\documentclass[a4paper, titlepage=false, parskip=full-, 10pt]{scrartcl}

\usepackage[utf8]{inputenc}
\usepackage[T1]{fontenc}
\usepackage[english, ngerman]{babel}
\usepackage{babelbib}
\usepackage{hyperref}
\usepackage{listings}
\usepackage{framed}
\usepackage{color}
\usepackage{graphicx}
\usepackage[normalem]{ulem}
\usepackage{cancel}
\usepackage{amsmath}
\usepackage{amssymb}
\usepackage{amsthm}
\usepackage{algorithm}
\usepackage{algorithmic}
\usepackage{geometry}
\usepackage{paralist}
\geometry{a4paper, top=20mm, left=35mm, right=25mm, bottom=40mm}

\newcounter{tasknbr}
\setcounter{tasknbr}{1}
\newenvironment{task}[1]{{\bf Aufgabe \arabic {tasknbr}\stepcounter{tasknbr}} (#1):\begin{enumerate}}{\end{enumerate}}
\newcommand{\subtask}[1]{\item[#1)]}

% Listings -----------------------------------------------------------------------------
\definecolor{red}{rgb}{.8,.1,.2}
\definecolor{blue}{rgb}{.2,.3,.7}
\definecolor{lightyellow}{rgb}{1.,1.,.97}
\definecolor{gray}{rgb}{.7,.7,.7}
\definecolor{darkgreen}{rgb}{0,.5,.1}
\definecolor{darkyellow}{rgb}{1.,.7,.3}
\lstloadlanguages{C++,[Objective]C,Java}
\lstset{
escapeinside={§§}{§§},
basicstyle=\ttfamily\footnotesize\mdseries,
columns=fullflexible, % typewriter font look better with fullflex
keywordstyle=\bfseries\color{blue},
% identifierstyle=\bfseries,
commentstyle=\color{darkgreen},      
stringstyle=\color{red},
numbers=left,
numberstyle=\ttfamily\scriptsize\color{gray},
% stepnumber=5,
% numberfirstline=true,
breaklines=true,
% prebreak=\\,
showstringspaces=false,
tabsize=4,
captionpos=b,
% framexrightmargin=-.2\textwidth,
float=htb,
frame=tb,
frameshape={RYR}{y}{y}{RYR},
rulecolor=\color{black},
xleftmargin=15pt,
xrightmargin=4pt,
aboveskip=\bigskipamount,
belowskip=\bigskipamount,
backgroundcolor=\color{lightyellow},
extendedchars=true,
belowcaptionskip=15pt}

%% Enter current values here: %%
\newcommand{\lecture}{Algorithmische Geometrie SS15}
\newcommand{\tutor}{}
\newcommand{\assignmentnbr}{1}
\newcommand{\students}{Julius Auer, Alexa Schlegel}
%%-------------------------------------%%

\begin{document}  
{\small \textsl{\lecture \hfill \tutor}}
\hrule
\begin{center}
\textbf{Übungsblatt \assignmentnbr}\\
[\bigskipamount]
{\small \students}
\end{center}
\hrule

\begin{task}{Geradendarstellung}

\subtask{a}
Eine Gerade \(g\) in der Ebene kann man auf verschiedene Art und Weise darstellen:

\begin{compactenum}[(a)]
	\item Als affine Hülle von zwei Punkten $p$ und $q$:\\
	$g = {\overrightarrow{p} + \lambda (\overrightarrow{q} - \overrightarrow{p}) | \lambda \in \mathbb{R}}$.
	\item Als Nullstellenmenge einer linearen Gleichung:\\
	$g = {\overrightarrow{x}=(x,y)|ax+by+c=0}$.
	\item In Hessischer Normalform (HNF):\\
	$g = {\overrightarrow{x}=\overrightarrow{n}\overrightarrow{x}+c=0}$ wobei $\|\overrightarrow{n}\| =1$.
\end{compactenum}

Geben Sie an, wie man zwischen den verschiedenen Darstellungen umrechnen kann, und beschreiben Sie die arithmetischen Operationen, die dafür benötigt werden.

[TODO]

\subtask{b}
Sei eine Gerade $g$ in HNF gegeben, $g={ \overrightarrow{x}\in\mathbb{R}^2|\overrightarrow{n}\overrightarrow{x}+c=0 }$. Zeigen Sie, dass für jeden Punkt $ p\in\mathbb{R}^2$ der Abstand von $p$ zu $g$, $d(p,g):=\textrm{main}{\|p-q\|,q\in g}$ gegeben ist durch $|\overrightarrow{n}\overrightarrow{p}+c|$.



Für Quellcode:\\
\lstset{language=Java}
\begin{lstlisting}
public class Main {
	public static void main(String[] args) {
		System.out.println("Hello World!");
	}
}
\end{lstlisting}

\end{task}
\end{document}