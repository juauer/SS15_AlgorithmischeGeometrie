% !TEX TS-program = pdflatex
% !TEX encoding = UTF-8 Unicode

\documentclass[a4paper, titlepage=false, parskip=full-, 10pt]{scrartcl}

\usepackage[utf8]{inputenc}
\usepackage[T1]{fontenc}
\usepackage[english, ngerman]{babel}
\usepackage{babelbib}
\usepackage{hyperref}
\usepackage{listings}
\usepackage{framed}
\usepackage{color}
\usepackage{graphicx}
\usepackage[normalem]{ulem}
\usepackage{cancel}
\usepackage{amsmath}
\usepackage{amssymb}
\usepackage{amsthm}
\usepackage{algorithm}
\usepackage{algorithmic}
\usepackage{geometry}
\usepackage{subfigure}
\geometry{a4paper, top=20mm, left=35mm, right=25mm, bottom=40mm}

\newcounter{tasknbr}
\setcounter{tasknbr}{1}
\newenvironment{task}[1]{{\bf Aufgabe \arabic {tasknbr}\stepcounter{tasknbr}} (#1):\begin{enumerate}}{\end{enumerate}}
\newcommand{\subtask}[1]{\item[#1)]}

% Listings -----------------------------------------------------------------------------
\definecolor{red}{rgb}{.8,.1,.2}
\definecolor{blue}{rgb}{.2,.3,.7}
\definecolor{lightyellow}{rgb}{1.,1.,.97}
\definecolor{gray}{rgb}{.7,.7,.7}
\definecolor{darkgreen}{rgb}{0,.5,.1}
\definecolor{darkyellow}{rgb}{1.,.7,.3}
\lstloadlanguages{C++,[Objective]C,Java}
\lstset{
escapeinside={§§}{§§},
basicstyle=\ttfamily\footnotesize\mdseries,
columns=fullflexible, % typewriter font look better with fullflex
keywordstyle=\bfseries\color{blue},
% identifierstyle=\bfseries,
commentstyle=\color{darkgreen},      
stringstyle=\color{red},
numbers=left,
numberstyle=\ttfamily\scriptsize\color{gray},
% stepnumber=5,
% numberfirstline=true,
breaklines=true,
% prebreak=\\,
showstringspaces=false,
tabsize=4,
captionpos=b,
% framexrightmargin=-.2\textwidth,
float=htb,
frame=tb,
frameshape={RYR}{y}{y}{RYR},
rulecolor=\color{black},
xleftmargin=15pt,
xrightmargin=4pt,
aboveskip=\bigskipamount,
belowskip=\bigskipamount,
backgroundcolor=\color{lightyellow},
extendedchars=true,
belowcaptionskip=15pt}

%% Enter current values here: %%
\newcommand{\lecture}{Algorithmische Geometrie SS15}
\newcommand{\tutor}{}
\newcommand{\assignmentnbr}{9}
\newcommand{\students}{Julius Auer, Alexa Schlegel}
%%-------------------------------------%%

\begin{document}  
{\small \textsl{\lecture \hfill \tutor}}
\hrule
\begin{center}
\textbf{Übungsblatt \assignmentnbr}\\
[\bigskipamount]
{\small \students}
\end{center}
\hrule

\begin{task}{Platonische Körper}
\item[]
Platonische Körper sind volkommen regelmäßige konvexe Polyeder. Polyeder sind dreidimensionale Körper, die von Polygonen (Vielecken) als Seitenflächen begrenzt sind.


\subtask{a}
Die Summe aller zusammentreffender Innenwinkel muss $< 360^\circ$ sein. Ist die Summe genau $360^\circ$ so entsteht eine Fläche in der Ebene, bei $>360^\circ$ ist die Ecke nicht mehr konvex.

Ein regelmäßiges Dreieck hat einen Innenwinkel von $60^\circ$, ein Viereck von $90^\circ$, ein Fünfeck von $108^\circ$, ein Sechseck von $120^\circ$. Ein Sechseck als Facette kann es damit nicht geben ($3\cdot120^\circ = 360^\circ$).

Somit kann es nur regelmäßige $m$-Ecke mit $m \in \{3,4,5\}$ geben.

An einer Ecke müssen $\geq 3$ Facetten zusammentreffen damit eine Ecke entsteht. So können höchstens 5 regelmäßige Dreiecke an einer Ecke zusammenstoßen ($360^\circ / 60^\circ = 6$), höchstens 3 regelmäßige Vierecke ($360^\circ / 90^\circ = 4$) und höchstens 3 regelmäßige Fünfecke ($360^\circ / 108^\circ = 3.3$). Damit ergibt sich für $k = \{3,4,5\}$, und die folgenden Kombinationen für $k$ und $m$:
$$m=3, k=3$$
$$m=3, k=4$$
$$m=3, k=5$$
$$m=4, k=3$$
$$m=5, k=3$$


\subtask{b}
Ikosaeder und Dodekaeder als geometrische Graphen 

\begin{figure}[htpb]
\begin{center}
\includegraphics[width=7cm]{iko}
\end{center}
\caption{Der Ikosaeder hat 20 Facetten (gleichseitige Dreiecken), 30 Kanten und 12 Ecken.}
\end{figure}

\begin{figure}[htpb]
\begin{center}
\includegraphics[width=7cm]{dode}
\end{center}
\caption{Der Dodekaeder hat 12 Facetten (regelmäßiges Fünfeck), 30 Kanten und 20 Ecken.}
\end{figure}




\end{task}


\begin{task}{d-dimensionale Polytope}
\item[]

$d$-dimensionaler Einheitswürfel $W_d$\\
\subtask{a}
Geben Sie die Ecken und die $d - 1$-dimensionalen Facetten von $W_d$ an.
Wieviele gibt es davon?
\subtask{b}
Zeichnen Sie den $W_4$ dh. seine Ecken und Kanten möglichst anschaulich.

$d$-dimensionaler Einheitssimplex $S_d$\\
\subtask{a}
Geben Sie die Ecken und die $d - 1$-dimensionalen Facetten von $S_d$ an
Wieviele gibt es davon?
\subtask{b}
Zeichnen Sie den $S_4$ dh. seine Ecken und Kanten möglichst anschaulich.


\end{task}

\begin{task}{Konvexe Hülle}
\item[]
Punkte einen nach dem anderen hinzuzufügen und konvexe Hülle aktualisieren klingt gut: Wenn der hinzukommende Punkt innerhalb bzw. auf dem Rand der kovexen Hülle liegt, dann muss man nichts tun, nur wenn er außerhalb liegt wird es interessant.

Von dem Punkt aus gesehen, würde ich das Ding in die Ebene projizieren. Flächen einfügen zwischen allen Kanten die auf dem entstanden Rand liegen und dem Punkt. Kanten zwischen allen Knoten auf Rand und Punkt hinzufügen

Der ganze Rest der nun verdeckt wird wegschmeißen.
\end{task}
\end{document}