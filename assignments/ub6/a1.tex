Wir betrachten als erstes Voronoi Kanten zwischen Punkt $P$ und Strecke $s$, diese bestehen grundsätzlich aus 3 Abschnitten. (i) und (iii) sind Strecken bzw. Strahlen. Es ist die Voronoi Kante zwischen dem Punkt $P$ und dem linken $A$ bzw. rechten Streckenende $B$. (ii) ist eine Parabel, deren Form ist abhängig vom Abstand zwischen $P$ und $s$. In welchem Abschnitt $P$ liegt ist egal, denn die Zusammensetzung der Voronoi Kante bleibt gleich.

\begin{figure}[h]
\begin{center}
\includegraphics[width=7cm]{img/punkt-strecke.png}
\end{center}
\caption{Bisektor Punkt und Strecke}
\label{fig:c1}
\end{figure}

Betrachten wir nun die Bisektoren von zwei Strecken $s$ und $t$, dabei unterscheiden wir in (a) parallele Strecken und (b) nicht parallele Strecken.

\paragraph*{(a): parallele Strecken:}

\begin{itemize}
\item (1) $s$ und $t$ liegen auf einer Geraden (Spezialfall)
\item (2) überschneiden sich nicht
\item (3) überschneiden sich zum Teil oder ganz
\end{itemize}

Im Fall (2) und (3) entstehen 5 Bereiche:\\
(i) Gerade: Bisektor von den linkesten Punkten $A$ und $C$\\
(ii) Parabel: $s$ und $C$\\
(iii) Gerade: Gerade dazwischen\\
(iv) Parabel: $t$ und $B$\\
(v) Gerade: Bisektor von den rechtesten Punkten $B$ und $D$\\

\begin{figure}[h]
\begin{center}
\includegraphics[width=7cm]{img/ssp1.png}
\end{center}
\caption{(1) Bisektor von zwei parallelen Geraden, auf einer Geraden.}
\label{fig:a1}
\end{figure}

\begin{figure}[h]
\begin{center}
\includegraphics[width=7cm]{img/ssp2.png}
\end{center}
\caption{(2) Bisektor von zwei parallelen Geraden, keine Überschneidung.}
\label{fig:a2}
\end{figure}

\begin{figure}[h]
\begin{center}
\includegraphics[width=7cm]{img/ss3.png}
\end{center}
\caption{(3) Bisektor von zwei parallelen Geraden, Überschneidung.}
\label{fig:a3}
\end{figure}

\begin{figure}[h]
\begin{center}
\includegraphics[width=7cm]{img/ss4.png}
\end{center}
\caption{(3) Bisektor von zwei parallelen Geraden, komplett innerhalb.}
\label{fig:a4}
\end{figure}

\paragraph*{(b): nicht parallele Strecken}

\begin{itemize}
\item (1) überschneiden sich nicht
\item (2) überschneiden sich zum Teil oder ganz
\end{itemize}

Hier entstehen im Fall (1) 4 Bereiche, im Falls (2) 7 Bereiche:\\

Erster Fall: wenn sie sich nicht überschneiden:\\
(i) Gerade\\
(ii) Winkelhalbierende\\
(iii) Parabel\\
(iv) Gerade\\

\begin{figure}[h]
\begin{center}
\includegraphics[width=7cm]{img/ssnpout.png}
\end{center}
\caption{(3) Bisektor von zwei nicht parallelen Geraden, keine Überschneidung.}
\label{fig:a5}
\end{figure}

Zweiter Fall: wenn sie sich überschneiden:\\
(i) Gerade\\
(ii) Parabel\\
(iii) Winkelhalbierende\\
(iv) Parabel\\
(v) Winkelhalbierende\\
(vi) Parabel\\
(vii) Gerade\\

\begin{figure}[h]
\begin{center}
\includegraphics[width=7cm]{img/ssnpin.png}
\end{center}
\caption{(3) Bisektor von zwei nicht parallelen Geraden, mit Überschneidung.}
\label{fig:a5}
\end{figure}


Anzahl der Ecken: vielleicht wenn ich die Ecken weiß dann kommt man auch auf die Kanten.
Anzahl der Kanten: hm das weiß ich nicht.
Anzahl der Zellen: ist gleich Anzahl der Strecken.

* Wie sehen die Voronoi-Kanten aus $\rightarrow$ Bilder von allen Fällen\\
* Aus wie vielen Ecken, Kanten und Zellen kann $VD(S)$ höchstens bestehen?\\
* Zeigen Sie dazu, dass die Voronoi-Regionen zusammenhängend sind.