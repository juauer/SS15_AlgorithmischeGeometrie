% !TEX TS-program = pdflatex
% !TEX encoding = UTF-8 Unicode

\documentclass[a4paper, titlepage=false, parskip=full-, 10pt]{scrartcl}

\usepackage[utf8]{inputenc}
\usepackage[T1]{fontenc}
\usepackage[english, ngerman]{babel}
\usepackage{babelbib}
\usepackage{hyperref}
\usepackage{listings}
\usepackage{framed}
\usepackage{color}
\usepackage{graphicx}
\usepackage[normalem]{ulem}
\usepackage{cancel}
\usepackage{amsmath}
\usepackage{amssymb}
\usepackage{amsthm}
\usepackage{algorithm}
\usepackage{algorithmic}
\usepackage{geometry}
\usepackage{subfigure}
\geometry{a4paper, top=20mm, left=35mm, right=25mm, bottom=40mm}

\newcounter{tasknbr}
\setcounter{tasknbr}{1}
\newenvironment{task}[1]{{\bf Aufgabe \arabic {tasknbr}\stepcounter{tasknbr}} (#1):\begin{enumerate}}{\end{enumerate}}
\newcommand{\subtask}[1]{\item[#1)]}

% Listings -----------------------------------------------------------------------------
\definecolor{red}{rgb}{.8,.1,.2}
\definecolor{blue}{rgb}{.2,.3,.7}
\definecolor{lightyellow}{rgb}{1.,1.,.97}
\definecolor{gray}{rgb}{.7,.7,.7}
\definecolor{darkgreen}{rgb}{0,.5,.1}
\definecolor{darkyellow}{rgb}{1.,.7,.3}
\lstloadlanguages{C++,[Objective]C,Java}
\lstset{
escapeinside={§§}{§§},
basicstyle=\ttfamily\footnotesize\mdseries,
columns=fullflexible, % typewriter font look better with fullflex
keywordstyle=\bfseries\color{blue},
% identifierstyle=\bfseries,
commentstyle=\color{darkgreen},      
stringstyle=\color{red},
numbers=left,
numberstyle=\ttfamily\scriptsize\color{gray},
% stepnumber=5,
% numberfirstline=true,
breaklines=true,
% prebreak=\\,
showstringspaces=false,
tabsize=4,
captionpos=b,
% framexrightmargin=-.2\textwidth,
float=htb,
frame=tb,
frameshape={RYR}{y}{y}{RYR},
rulecolor=\color{black},
xleftmargin=15pt,
xrightmargin=4pt,
aboveskip=\bigskipamount,
belowskip=\bigskipamount,
backgroundcolor=\color{lightyellow},
extendedchars=true,
belowcaptionskip=15pt}

%% Enter current values here: %%
\newcommand{\lecture}{Algorithmische Geometrie SS15}
\newcommand{\tutor}{}
\newcommand{\assignmentnbr}{10}
\newcommand{\students}{Julius Auer, Alexa Schlegel}
%%-------------------------------------%%

\begin{document}  
{\small \textsl{\lecture \hfill \tutor}}
\hrule
\begin{center}
\textbf{Übungsblatt \assignmentnbr}\\
[\bigskipamount]
{\small \students}
\end{center}
\hrule

\begin{task}{Wieviele Punkte liegen auf einer Geraden?}
\item[]
* effizienten Algorithmus an, der für eine Menge $n$ Punkten ${p_1, p_2, \dots, p_n} \in \mathbb{R}$ die maximale Zahl dieser Punkte bestimmt, die auf einer Geraden liegen\\
* Hinweis: Dualisierung. $O(n^2 \log n$) ist akzeptabel aber noch nicht optimal\\

Total ganz naiver Ansatz:\\
* stelle alle möglichen Geraden auf, das sind wie viele? bestimmt: $O(n^2)$\\
* pro Gerade schaue wie viele Punkte drauf liegen, d.h teste für jeden Punkt und jede Gerade (oha das ist viel) $O(n^2 \cdot n) = O(n^3)$\\
* und dann das maximum nehmen, hm.\\

Dualisierung - Was ist das?\\
$$p=(p_x,p_y) \rightarrow p^* : b = p_x \cdot a - p_y$$
$$l : y = m \cdot x + c \rightarrow l^* = (m, -c)$$

Wenn $p$ liegt auf $l$, dann $p^*$ liegt auf $l^*$\\
Punkte $q, r, s$ sind kolinear, dann $q^*, r^*, s^*$ schneiden sich in gemeinsamen Punkt.\\

Was macht man damit? Man dualisiert alle Punkte ${p_1^*, \dots, p_n^*}$ und schaut wo sich die meisten Geraden schneiden? Die Anzahl der Geraden ist dann das, was wir wohl gesucht haben.\\
Jetzt muss man erstmal alle Schnittpunkte finden, oder?\\

TODO\\

\end{task}

\begin{task}{Geben Sie für beliebiges $n \in \mathbb{N}$ eine Punktmenge der Größe $n \in \mathbb{R}^4$ an, deren konvexe Hülle die Größe (=Anzahl der Facetten) $\Omega(n^2)$ hat.}
\item[]

* 4D Punktmenge\\
* brauch ich wohl mindestens 5 Punkte damit das Sinn macht\\
* kann man das nicht auch irgendwie mit der Dualisierung umdrehen, sodass man eigentlich (anstelle der Facetten), die Anzahl der Ecken oder sowas in $\Omega(n^2)$ sucht?\\

TODO\\

\end{task}

\begin{task}{inkementelle Kontruktion}
\item[]

Eingaben, Einfügereihenfolgen, sodass Laufzeit $\Omega(n^2)$\\

\subtask{a}
konvexe Hülle einer Punktmenge in $\mathbb{R}^3$\\

TODO\\

\subtask{b}
Trapezzerlegung eines Arrangements von Strecken im $\mathbb{R}^2$\\

TODO\\

\end{task}

\end{document}