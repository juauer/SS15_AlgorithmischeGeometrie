% !TEX TS-program = pdflatex
% !TEX encoding = UTF-8 Unicode

\documentclass[a4paper, titlepage=false, parskip=full-, 10pt]{scrartcl}

\usepackage[utf8]{inputenc}
\usepackage[T1]{fontenc}
\usepackage[english, ngerman]{babel}
\usepackage{babelbib}
\usepackage{hyperref}
\usepackage{listings}
\usepackage{framed}
\usepackage{color}
\usepackage{graphicx}
\usepackage[normalem]{ulem}
\usepackage{cancel}
\usepackage{amsmath}
\usepackage{amssymb}
\usepackage{amsthm}
\usepackage{algorithm}
\usepackage{algorithmic}
\usepackage{geometry}
\usepackage{subfigure}
\geometry{a4paper, top=20mm, left=35mm, right=25mm, bottom=40mm}

\newcounter{tasknbr}
\setcounter{tasknbr}{1}
\newenvironment{task}[1]{{\bf Aufgabe \arabic {tasknbr}\stepcounter{tasknbr}} (#1):\begin{enumerate}}{\end{enumerate}}
\newcommand{\subtask}[1]{\item[#1)]}

% Listings -----------------------------------------------------------------------------
\definecolor{red}{rgb}{.8,.1,.2}
\definecolor{blue}{rgb}{.2,.3,.7}
\definecolor{lightyellow}{rgb}{1.,1.,.97}
\definecolor{gray}{rgb}{.7,.7,.7}
\definecolor{darkgreen}{rgb}{0,.5,.1}
\definecolor{darkyellow}{rgb}{1.,.7,.3}
\lstloadlanguages{C++,[Objective]C,Java}
\lstset{
escapeinside={§§}{§§},
basicstyle=\ttfamily\footnotesize\mdseries,
columns=fullflexible, % typewriter font look better with fullflex
keywordstyle=\bfseries\color{blue},
% identifierstyle=\bfseries,
commentstyle=\color{darkgreen},      
stringstyle=\color{red},
numbers=left,
numberstyle=\ttfamily\scriptsize\color{gray},
% stepnumber=5,
% numberfirstline=true,
breaklines=true,
% prebreak=\\,
showstringspaces=false,
tabsize=4,
captionpos=b,
% framexrightmargin=-.2\textwidth,
float=htb,
frame=tb,
frameshape={RYR}{y}{y}{RYR},
rulecolor=\color{black},
xleftmargin=15pt,
xrightmargin=4pt,
aboveskip=\bigskipamount,
belowskip=\bigskipamount,
backgroundcolor=\color{lightyellow},
extendedchars=true,
belowcaptionskip=15pt}

%% Enter current values here: %%
\newcommand{\lecture}{Algorithmische Geometrie SS15}
\newcommand{\tutor}{}
\newcommand{\assignmentnbr}{10}
\newcommand{\students}{Julius Auer, Alexa Schlegel}
%%-------------------------------------%%

\begin{document}  
{\small \textsl{\lecture \hfill \tutor}}
\hrule
\begin{center}
\textbf{Übungsblatt \assignmentnbr}\\
[\bigskipamount]
{\small \students}
\end{center}
\hrule

\begin{task}{}
\item[]
\begin{itemize}
\item Eingabe: Menge von $n$ Punkten $P=\{(x_1,y_1),...,(x_n,y_n)\}$
\item Für jedes Paar von Punkten $((x_i,y_i),(x_j,y_j))$ mit $i<j$ interpretiere die Punkte als Geraden $g_1=x_i\cdot x-y_i$ und $g_2=x_j\cdot x-y_j$ und finde den Schnittpunkt $s$ zwischen $g_1$ und $g_2$. Die duale Gerade zu $s$ ist nun genau die Gerade auf der $(x_i,y_i)$ und $(x_j,y_j)$ liegen.
\item Speichere in einer geeigneten Datenstruktur, wie oft ein und derselbe Schnittpunkt bei unterschiedlichen Punkt-Paaren gefunden wurde - der größte Wert repräsentiert schlußendlich die Gerade (im Dualen) auf der die meisten Punkte liegen.
\end{itemize}

Als einfache Datenstruktur zum Ablegen der Schnittpunkte können binäre Suchbäume o.ä. verwendet werden - bei $n^2$ vielen Punkten-Paaren benötigt man für das Einfügen der Schnittpunkte in die Datenstruktur insgesamt $O(n^2\cdot\log n)$ Zeit.

Der $\log n$-Faktor lässt sich einsparen, wenn in konstanter Zeit verwaltet werden kann, wie oft ein Schnittpunkt errechnet wurde. Da die maximale Anzahl an Schnittpunkten bekannt ist ($n^2$) kann hier bei Verwendung einer Hashtabelle - die $O(n^2)$ Platz benötigt - perfektes Hashing garantiert werden. Schließlich wird für jeden Key nur ein Zähler inkrementiert (es sind also keine Listen erforderlich) und die maximale Anzahl Keys steht vorher fest.

Zum Einfügen wird so nur $O(1)$ Zeit benötigt und der $\log n$-Faktor entfällt. Als Hashfunktion können die x- und y-Koordinaten der Punkte z.B. als String interpretiert und konkateniert werden. Ggf. sollte allerdings ein gewisser Aufwand betrieben werden um Rundungsfehler sinnvoll zu behandeln ^^
\end{task}

\begin{task}{}
\item[]
Die größtmögliche Anzahl an Ecken und Facetten der konvexen Hülle einer Punktmenge $P$ mit $|P|=n$ findet man bei zyklischen Polytopen. Ein passendes Polytop kann z.B. die Form:
$$P=\{m(1),...,m(n)\}$$
haben, wobei $m:\mathbb{R}\rightarrow\mathbb{R}^4$ eine Momentfunktion ist:
$$m(x)=(x,x^2,x^3,x^4)$$
Keine $d+1$ Punkte aus $P$ sind affin abhängig und alle Facetten somit Simplices. Jede 2-Teilmenge aus $P$ bestimmt ein solches Simplex. Es lassen sich $\begin{pmatrix}n\\2\end{pmatrix}\in\Omega (n^2)$ solcher Teilmengen bilden.
\end{task}

\begin{task}{inkementelle Kontruktion}
\item[]

\subtask{a}
Gegeben ist der folgende Algorithmus zur Berechnung der konvexen Hülle $CH(P)$ einer gegeben $n$-elementige Punktmenge $P=\{p_1,\dots, p_n\}$ in $\mathbb{R}^3$.

(1) Initialisiere mit $CH(p_1, p_2, p_3, p_4)$\\
(2) Füge Punkte nacheinander hinzu und aktualisiere die konveke Hülle in jedem Schritt: Für einen Punkt $p_i$ der einzufügen ist, wird $CH(p_1, \dotsm, p_{i-1})$ von $p_i$ aus gesehen in die Ebene projiziert und der Rand bestimmt. $p_i$ wird nun mit allen Punkten auf dem Rand verbunden, das sind die neuen Kanten der CH, Facetten werden neu bestimmt. Alle Facetten und Kanten die jetzt verdeckt werden, werden gelöscht.

Wir müssen nun eine Menge von Punkten finden, die alle zur konvexen Hülle gehören werden. In jedem Schritt $i$ muss also der Rand aus allen Punkten der konvexen Hülle bestehen (also aus $p_1, \dots, p_{i-1}$. Das hat zur Folge, dass in jedem Schritt alle $i-2$ Kanten entfernt werden und $i-1$ Kanten hinzugefügt werden.

Hier bietet sich wie in Aufgabe 2 die folgende Momentfunktion an:
$$m(x)=(x,x^2,x^3)$$

\emph{Eingabe:} $P=\{(1,1,1), (2,4,8), (3,9,27), \dots, (n,n^2, n^3)\}$


\emph{Einfügereihenfolgen:} aufsteigend


\emph{Laufzeit:} $\sum\limits_{i=1}^n O(i) = \frac{n^2 + n}{2} = O(n^2)$

\subtask{b}
Trapezzerlegung eines Arrangements von Strecken im $\mathbb{R}^2$\\

TODO\\

\end{task}
\end{document}