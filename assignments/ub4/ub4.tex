% !TEX TS-program = pdflatex
% !TEX encoding = UTF-8 Unicode

\documentclass[a4paper, titlepage=false, parskip=full-, 10pt]{scrartcl}

\usepackage[utf8]{inputenc}
\usepackage[T1]{fontenc}
\usepackage[english, ngerman]{babel}
\usepackage{babelbib}
\usepackage{hyperref}
\usepackage{listings}
\usepackage{framed}
\usepackage{color}
\usepackage{graphicx}
\usepackage[normalem]{ulem}
\usepackage{cancel}
\usepackage{amsmath}
\usepackage{amssymb}
\usepackage{amsthm}
\usepackage{algorithm}
\usepackage{algorithmic}
\usepackage{geometry}
\usepackage{subfigure}
\geometry{a4paper, top=20mm, left=35mm, right=25mm, bottom=40mm}

\newcounter{tasknbr}
\setcounter{tasknbr}{1}
\newenvironment{task}[1]{{\bf Aufgabe \arabic {tasknbr}\stepcounter{tasknbr}} (#1):\begin{enumerate}}{\end{enumerate}}
\newcommand{\subtask}[1]{\item[#1)]}

% Listings -----------------------------------------------------------------------------
\definecolor{red}{rgb}{.8,.1,.2}
\definecolor{blue}{rgb}{.2,.3,.7}
\definecolor{lightyellow}{rgb}{1.,1.,.97}
\definecolor{gray}{rgb}{.7,.7,.7}
\definecolor{darkgreen}{rgb}{0,.5,.1}
\definecolor{darkyellow}{rgb}{1.,.7,.3}
\lstloadlanguages{C++,[Objective]C,Java}
\lstset{
escapeinside={§§}{§§},
basicstyle=\ttfamily\footnotesize\mdseries,
columns=fullflexible, % typewriter font look better with fullflex
keywordstyle=\bfseries\color{blue},
% identifierstyle=\bfseries,
commentstyle=\color{darkgreen},      
stringstyle=\color{red},
numbers=left,
numberstyle=\ttfamily\scriptsize\color{gray},
% stepnumber=5,
% numberfirstline=true,
breaklines=true,
% prebreak=\\,
showstringspaces=false,
tabsize=4,
captionpos=b,
% framexrightmargin=-.2\textwidth,
float=htb,
frame=tb,
frameshape={RYR}{y}{y}{RYR},
rulecolor=\color{black},
xleftmargin=15pt,
xrightmargin=4pt,
aboveskip=\bigskipamount,
belowskip=\bigskipamount,
backgroundcolor=\color{lightyellow},
extendedchars=true,
belowcaptionskip=15pt}

%% Enter current values here: %%
\newcommand{\lecture}{Algorithmische Geometrie SS15}
\newcommand{\tutor}{}
\newcommand{\assignmentnbr}{4}
\newcommand{\students}{Julius Auer, Alexa Schlegel}
%%-------------------------------------%%

\begin{document}  
{\small \textsl{\lecture \hfill \tutor}}
\hrule
\begin{center}
\textbf{Übungsblatt \assignmentnbr}\\
[\bigskipamount]
{\small \students}
\end{center}
\hrule

\begin{task}{Bewegungsplanung in der Ebene}\item[]
Idee: Aus H Voronoi-Diagramm basteln. Aus dem Graph die Kanten entfernen, wo der Roboter nicht durchpasst. {\bf Auf} den verbliebenen Voronoi-Kanten kann sich der Roboter dann bewegen.

- VD basteln\\
- prüfen, ob s und z nicht bereits kollidieren\\
- Kanten rausschmeissen, wo der Abstand zwischen den Punkten der VRs kleiner ist als Rs Durchmesser\\
- VRs suchen, in denen s und z liegen\\
- Kante suchen die von s aus erreichbar ist, ohne mit dem Hindernis in dieser VR zu kollidieren (Lot fällen?)\\
- Weg suchen über Knoten entlang Kanten (Dijkstra)
\end{task}

\begin{task}{Geometriche Graphen}
\subtask{a}
Induktion über $|V|$:

I.A.: Für $G_0=(V,E,F)$ zusammenhängend und planar mit $|V|=1$ folgt offensichtlich $|E|=0$ und $|F|=1$.

I.V.: Für $G_n=(V,E,F)$ zusammenhängend und planar mit $|V|=n$ gilt: $|E|=|V|+|F|-2$

$n\rightarrow n+1$: Aus $G_{n+1}=(V',E',F')$ zusammenhängend und planar mit $|V'|=n+1$ werden ein Knoten $v$ und alle zu $v$ inzidenten Kanten entfernt.

Fall 1: Der entstehende Graph ist zusammenhängend:\\
Für den entstehenden Graphen $G_n=(V,E,F)$ gilt $|V|=n$ und $|E|=|E'|-deg(v)$ sowie die I.V.. Nun muss $v$ in einer Facette $f$ von $G_n$ liegen und somit jede zu $v$ inzidente Kante in $G_{n+1}$ zu einem Knoten auf dem Rand von $f$ führen (sonst wäre Planarität verletzt). Die zu $v$ inzidenten Kanten unterteilen $f$ offensichtlich in $deg(v)-1$ zusätzliche Facetten. Somit gilt für $G_{n+1}$:
\begin{align*}
F'&=F+deg(v)-1\\
&=|E|-|V|+2+deg(v)-1\\
&=|E'|-deg(v)-|V'|+1+2+deg(v)-1\\
&=|E'|-|V'|+2
\end{align*}

Fall 2: Der entstehende Graph ist nicht zusammenhängend:\\
Dann entstehen $k\le deg(v)$ planare Zusammenhangskomponenten $G^1,...,G^k$. Für jede gilt die I.V..\\
Für das Verbinden dieser ZHK waren zuvor $k$ Kanten erforderlich, die keine Facetten bilden (sonst hätte es einen Kreis und nicht mehrere ZHK gegeben). Die anderen $deg(v)-k$ Kanten bildeten $deg(v)-k$ Facetten (analog zur Argumentation in Fall 1).\\
Ferner ist klar, dass sich die $k$ ZHK eine Facette (die ''Äußere'') teilen - also $k-1$ Facetten zuviel gezählt werden.\\
Somit ergibt sich insgesamt für $|F'|$:
\begin{align*}
F'&=\sum_{i=0}^k|F^i|+(deg(v)-k)-(k-1)\\
&=\sum_{i=0}^k(|E^i|-|V^i|+2)+deg(v)-2\cdot k+1\\
&=|E'|-deg(v)-(|V'|-1)+2\cdot k+deg(v)-2\cdot k+1\\
&=|E'|-|V'|+2\\
\end{align*}
\qed

\subtask{b}
PENDING
\end{task}

\begin{task}{Voronoi-Diagramm}\item[]
- Suche in n einen Extrempunkt (minx, miny, o.ä.)\\
- suche in n zugehörige VR\\
- wiederholen (bis wieder an Startpunkt): suche in konstant (ggf. zeigen) kante die gemeinsamen Knoten mit einer der 'Strahl-Kanten' hat und die kleinste Drehung beschreibt (zur Strahl-Kante)

Schneller als nlogn geht nicht, sonst auch CH schneller und somit auch - wie in U3 gezeigt - Sortieren schneller. Für Sortieren ist aber nlogn optimal.
\end{task}
\end{document}