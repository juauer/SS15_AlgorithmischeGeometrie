% !TEX TS-program = pdflatex
% !TEX encoding = UTF-8 Unicode

\documentclass[a4paper, titlepage=false, parskip=full-, 10pt]{scrartcl}

\usepackage[utf8]{inputenc}
\usepackage[T1]{fontenc}
\usepackage[english, ngerman]{babel}
\usepackage{babelbib}
\usepackage{hyperref}
\usepackage{listings}
\usepackage{framed}
\usepackage{color}
\usepackage{graphicx}
\usepackage[normalem]{ulem}
\usepackage{cancel}
\usepackage{amsmath}
\usepackage{amssymb}
\usepackage{amsthm}
\usepackage{algorithm}
\usepackage{algorithmic}
\usepackage{geometry}
\usepackage{subfigure}
\geometry{a4paper, top=20mm, left=35mm, right=25mm, bottom=40mm}

\newcounter{tasknbr}
\setcounter{tasknbr}{1}
\newenvironment{task}[1]{{\bf Aufgabe \arabic {tasknbr}\stepcounter{tasknbr}} (#1):\begin{enumerate}}{\end{enumerate}}
\newcommand{\subtask}[1]{\item[#1)]}

% Listings -----------------------------------------------------------------------------
\definecolor{red}{rgb}{.8,.1,.2}
\definecolor{blue}{rgb}{.2,.3,.7}
\definecolor{lightyellow}{rgb}{1.,1.,.97}
\definecolor{gray}{rgb}{.7,.7,.7}
\definecolor{darkgreen}{rgb}{0,.5,.1}
\definecolor{darkyellow}{rgb}{1.,.7,.3}
\lstloadlanguages{C++,[Objective]C,Java}
\lstset{
escapeinside={§§}{§§},
basicstyle=\ttfamily\footnotesize\mdseries,
columns=fullflexible, % typewriter font look better with fullflex
keywordstyle=\bfseries\color{blue},
% identifierstyle=\bfseries,
commentstyle=\color{darkgreen},      
stringstyle=\color{red},
numbers=left,
numberstyle=\ttfamily\scriptsize\color{gray},
% stepnumber=5,
% numberfirstline=true,
breaklines=true,
% prebreak=\\,
showstringspaces=false,
tabsize=4,
captionpos=b,
% framexrightmargin=-.2\textwidth,
float=htb,
frame=tb,
frameshape={RYR}{y}{y}{RYR},
rulecolor=\color{black},
xleftmargin=15pt,
xrightmargin=4pt,
aboveskip=\bigskipamount,
belowskip=\bigskipamount,
backgroundcolor=\color{lightyellow},
extendedchars=true,
belowcaptionskip=15pt}

%% Enter current values here: %%
\newcommand{\lecture}{Algorithmische Geometrie SS15}
\newcommand{\tutor}{}
\newcommand{\assignmentnbr}{8}
\newcommand{\students}{Julius Auer, Alexa Schlegel}
%%-------------------------------------%%

\begin{document}  
{\small \textsl{\lecture \hfill \tutor}}
\hrule
\begin{center}
\textbf{Übungsblatt \assignmentnbr}\\
[\bigskipamount]
{\small \students}
\end{center}
\hrule

\begin{task}{Vorverarbeitungszeit für Bereichsbäume}
\item[]
Der Algorithmus zur Konstruktion - der $T(n)$ Zeit benötigt - ist straight-forward:

(1) Lege Knoten an und konstruiere sekundäre Struktur - im 2D-Fall ist die sekundäre Struktur ein ''einfacher'' binärer Baum mit $n$ Knoten, der in $O(n\cdot\log n)$ Zeit konstruiert werden kann.\\
(2) Finde Median der Punkte und teile deren Menge in zwei möglichst gleich große Teile. Mit geeignetem Median-Algorithmus (z.B. BFPRT) ist das in $O(n)$ möglich.\\
(3) Konstruiere die beiden resultierenden Teilbäume rekursiv in $O(T(\lfloor\frac{n}{2}\rfloor)+T(\lceil\frac{n}{2}\rceil))$.

Der Einfachheit halber sei $n$ im Folgenden eine 2er-Potenz. Dass
$$T(1)=1$$
ist klar. Somit ergibt sich für die Laufzeit: 

\begin{align*}
T(n)&=\overbrace{2\cdot T\left(\frac{n}{2}\right)}^{(3)}+\overbrace{n\cdot\log n}^{(1)}+\overbrace{n}^{(2)}\\
&=2\cdot T\left(\frac{n}{2}\right)+n\cdot (\log n+1)\\
&=2\cdot\left(2\cdot T\left(\frac{n}{4}\right)+\frac{n}{2}\cdot\left(\log\frac{n}{2}+1\right)\right)+n\cdot (\log n+1)\\
&=2\cdot\left(2\cdot\left(2\cdot T\left(\frac{n}{8}\right)+\frac{n}{4}\cdot\left(\log\frac{n}{4}+1\right)\right)+\frac{n}{2}\cdot\left(\log\frac{n}{2}+1\right)\right)+n\cdot (\log n+1)\\
&...\\
&=2^k\cdot T\left(\frac{n}{2^k}\right)+n\cdot\sum_{i=0}^{k-1}\left(\log\frac{n}{2^i}+1\right)
\end{align*}

Die Primärstruktur hat natürlich eine Höhe von maximal $\log n$:

\begin{align*}
T(n)&=2^{\log n}\cdot T\left(\frac{n}{2^{\log n}}\right)+n\cdot\sum_{i=0}^{\log n-1}\left(\log\frac{n}{2^i}+1\right)\\
&=n+n\cdot\left(\log\frac{n}{n-1}+\log n-1\right)\\
&=n\cdot\log\frac{n}{n-1}+n\cdot\log n\\
&\in n\cdot O(1)+n\cdot\log n\\
&\in O(n\cdot\log n)
\end{align*}
\qed
\end{task}

\begin{task}{dynamische Segmentsbäume}
\item[]

\end{task}

\begin{task}{Punkt-Rechteck-Anfragen}
\item[]

\end{task}
\end{document}